%%%%%%%%%%%%%%%%%
% This is an sample CV template created using altacv.cls
% (v1.1.5, 1 December 2018) written by LianTze Lim (liantze@gmail.com). Now compiles with pdfLaTeX, XeLaTeX and LuaLaTeX.
%
%% It may be distributed and/or modified under the
%% conditions of the LaTeX Project Public License, either version 1.3
%% of this license or (at your option) any later version.
%% The latest version of this license is in
%%    http://www.latex-project.org/lppl.txt
%% and version 1.3 or later is part of all distributions of LaTeX
%% version 2003/12/01 or later.
%%%%%%%%%%%%%%%%

%% If you need to pass whatever options to xcolor
\PassOptionsToPackage{dvipsnames}{xcolor}

%% If you are using \orcid or academicons
%% icons, make sure you have the academicons
%% option here, and compile with XeLaTeX
%% or LuaLaTeX.
% \documentclass[10pt,a4paper,academicons]{altacv}

%% Use the "normalphoto" option if you want a normal photo instead of cropped to a circle
% \documentclass[10pt,a4paper,normalphoto]{altacv}

\documentclass[10pt,a4paper,ragged2e]{altacv}

%% AltaCV uses the fontawesome and academicon fonts
%% and packages.
%% See texdoc.net/pkg/fontawecome and http://texdoc.net/pkg/academicons for full list of symbols. You MUST compile with XeLaTeX or LuaLaTeX if you want to use academicons.

% Change the page layout if you need to
\geometry{left=1cm,right=9cm,marginparwidth=6.8cm,marginparsep=1.2cm,top=1.25cm,bottom=1.25cm}

% Change the font if you want to, depending on whether
% you're using pdflatex or xelatex/lualatex
\ifxetexorluatex
  % If using xelatex or lualatex:
  \setmainfont{Carlito}
\else
  % If using pdflatex:
  \usepackage[utf8]{inputenc}
  \usepackage[T1]{fontenc}
  \usepackage[default]{lato}
\fi

\usepackage{hyperref}

% Change the colours if you want to
\definecolor{Mulberry}{HTML}{72243D}
\definecolor{SlateGrey}{HTML}{2E2E2E}
\definecolor{LightGrey}{HTML}{666666}
\colorlet{heading}{Sepia}
\colorlet{accent}{Mulberry}
\colorlet{emphasis}{SlateGrey}
\colorlet{body}{LightGrey}

% Change the bullets for itemize and rating marker
% for \cvskill if you want to
\renewcommand{\itemmarker}{{\small\textbullet}}
\renewcommand{\ratingmarker}{\faCircle}

%% sample.bib contains your publications
\addbibresource{sample.bib}

\begin{document}
\name{Marco Denisi}
\tagline{Software Engineer}
%\photo{2.8cm}{}
\personalinfo{%
  % Not all of these are required!
  % You can add your own with \printinfo{symbol}{detail}
  \email{marcodenisi35@gmail.com}
  \phone{+39 333 1364750}
  %\mailaddress{Via Don Rodrigo 7, Lecco, 23900, LC}
  \homepage{www.marcodenisi.dev}
  \linkedin{linkedin.com/in/marco-denisi}
  \github{github.com/marcodenisi}
  %% You MUST add the academicons option to \documentclass, then compile with LuaLaTeX or XeLaTeX, if you want to use \orcid or other academicons commands.
  % \orcid{orcid.org/0000-0000-0000-0000}
}

%% Make the header extend all the way to the right, if you want.
\begin{fullwidth}
\makecvheader
\end{fullwidth}

%% Depending on your tastes, you may want to make fonts of itemize environments slightly smaller
% \AtBeginEnvironment{itemize}{\small}

%% Provide the file name containing the sidebar contents as an optional parameter to \cvsection.
%% You can always just use \marginpar{...} if you do
%% not need to align the top of the contents to any
%% \cvsection title in the "main" bar.
\cvsection[page1sidebar]{Experience}

\cvevent{Senior Software Engineer}{Klarna}{Jul 2020 -- Ongoing}{Milan}
\begin{itemize}
\item Improved credit underwriting rule creation providing a visual builder using VueJS and Python
\item \textbf{Lead} the Italian Engineering Community
\item \textbf{Interviewed} 50+ candidates
\end{itemize}

\divider

\cvevent{Senior Software Engineer}{Sky Italia}{Jan 2019 -- Jul 2020}{Milan}
\begin{itemize}
\item Designed REST APIs using API-first approach with Swagger
\item Created \textbf{lambda function} to collect and send logs to AWS Cloudwatch
\item Took ownership of a new \textbf{SonarQube} flow
\end{itemize}

\divider

\cvevent{Software Engineer}{Satispay}{Jan 2018 -- Dec 2018}{Milan}
\begin{itemize}
\item Introduced \textbf{Flyway} to put database schema under version control
\item Implemented personal loans as a standalone \textbf{microservice}
\item Used \textbf{Terraform} to automate the cloud infrastructure creation
\end{itemize}
\divider

\cvevent{Software Engineer}{Open Reply}{May 2015 -- Dec 2017}{Milan/London}
\begin{itemize}
\item Engineered a report engine taking advantage of the Drools rule engine
\item Redesigned \textbf{automation tests framework} reducing regression tests development time
\item Boosted performances by introducing indexing with Apache Solr
\end{itemize}

\divider

\cvevent{Software Engineer}{Advanced Business Solutions SRL}{Oct 2014 -- Apr 2015}{Milan}
%\begin{itemize}
%\item Developed SOAP web services using Axis2
%\item Worked on CNHI Oracle Transportation Management customization using Oracle ADF
%\item Created PL/SQL procedures and functions to implement new business workflows
%\end{itemize}

\cvsection{Projects}

\cvevent{Gambe.ro}{\github{github.com/gambe-ro/lobsters-twitter-bot}}{}{}
I'm an active contributor to the \textbf{\href{http://gambe.ro}{gambe.ro}} community. Currently, I've contributed to the Twitter and Telegram bots. Bots are written in Python.

\divider

\medskip

\clearpage

%% If the NEXT page doesn't start with a \cvsection but you'd
%% still like to add a sidebar, then use this command on THIS
%% page to add it. The optional argument lets you pull up the
%% sidebar a bit so that it looks aligned with the top of the
%% main column.
% \addnextpagesidebar[-1ex]{page3sidebar}


\end{document}
